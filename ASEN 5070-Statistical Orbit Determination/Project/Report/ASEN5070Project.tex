\documentclass[12pt,a4paper,oneside]{article}

%################################# Preamble ##################################%

	%%%%%%%%%%%%%%%%%  << MATLAB INCLUSION>>  %%%%%%%%%%%%%%%%%
	\usepackage[]{mcode}
	% *mcode is in 
	%		/Users/zachdischner/Library/texmf/tex/latex/local
	% added to database with:
	%    		>>> sudo texhash
	%%%%%%%%%%%%%%%%%  << MATLAB INCLUSION>>  %%%%%%%%%%%%%%%%%
	
	%%%%%%%%%%%%%%%%%  << Margins and Spacing>>  %%%%%%%%%%%%%%%%%
	\usepackage[margin=0.75in]{geometry}
	%%%%%%%%%%%%%%%%%  << Margins and Spacing>>  %%%%%%%%%%%%%%%%%
	
	%%%%%%%%%%%%%%%%%  << IMAGE INCLUSION>>  %%%%%%%%%%%%%%%%%
	\usepackage{graphicx}
	%%%%%%%%%%%%%%%%%  << IMAGE INCLUSION>>  %%%%%%%%%%%%%%%%%
	\usepackage{natbib}
	\usepackage{color}
	\usepackage{cleveref}
	\usepackage{amsmath}
	\usepackage{graphicx}
	\usepackage{color}
	\usepackage{appendix}
	\usepackage{url}
	
	\usepackage{float}
	%\floatstyle{boxed} 		% Box around figures and things
	\restylefloat{figure}
	\setlength{\textfloatsep}{15pt plus 1.0pt minus 2.0pt}
	\setlength{\floatsep}{5pt plus 1.0pt minus 2.0pt}
	
	% Other packages
	%\usepackage{times, rawfonts, geometry}
	%\usepackage{amsmath,amssymb}
	%\usepackage{float}
	
	% New commands
	\newcommand{\ignore}[1]{}  % {} empty inside = %% comment
	
	% Scientific notation:
	\providecommand{\e}[1]{\ensuremath{\times 10^{#1}}}
	
	% General
	\newcommand{\parens} [1] {\left(  #1  \right)}
	\newcommand{\brackets} [1] {\left[ #1 \right]}
	\newcommand{\rootdir}{./Figures/}
	
	% Array
	\newcommand{\arrayp}[2]{\parens{ \begin{array}{#1}  #2 \end{array} } }
	\newcommand{\arrayb}[2]{\brackets{ \begin{array}{#1}  #2 \end{array} } }
	% Use like:  where {c|c} is a vertical bar, inserts into the matrix. 
	% $\arrayb{c|c}{y & 4\\y & 4\\y & X\\y & nn}$
	
	%Figure {HERE}   - use like:  \fig{figurename.extension}{Caption}{Label}
	\newcommand{\fig}[3]{
			\begin{center}
				\begin{figure}[H]
					\includegraphics[width=.9\textwidth]{\rootdir #1}
					\caption{#2}
					\label{#3}
				\end{figure}
			\end{center}
			}
	

%################################# Preamble ##################################%

\begin{document}

\title{ASEN 5070-Stastistical Orbit Determination-Final Project Report}
\author{Zach Dischner \\ University of Colorado at Boulder \\ Department of Aerospace Engineering}
\date{12-2-2012}
\maketitle

\begin{center}
	\begin{figure}[H]
		\includegraphics[width=.9\textwidth]{\rootdir GPS.jpg}    \cite{GPS}
	\end{figure}
\end{center}


\newpage
% Introduction
%----------------------------------------------------------------------------------------------------------------------------------------
\section{Introduction}
This report summarizes an investigation of various methods of statistical orbit determination, as outlined in ASEN 5070. All programming was performed in Matlab, using a combination of built in functions,  self-defined functions, and ones created in collaboration with others. I will examine the results and implications of various filter methodologies including:

\begin{itemize}
	\renewcommand{\labelitemi}{$\bullet$}
	\item Batch Processor
	\item Conventional Kalman (Sequential) Filter
	\item Extended Kalman Filter
	\item State Noise Compensation
	\item Alternative Methods for Determining \emph{P}, the Covariance Matrix
\end{itemize}
%----------------------------------------------------------------------------------------------------------------------------------------


\newpage
% Table Of Contents
%----------------------------------------------------------------------------------------------------------------------------------------
\tableofcontents{}
\listoffigures{}
%----------------------------------------------------------------------------------------------------------------------------------------


\newpage
% Background
%----------------------------------------------------------------------------------------------------------------------------------------
\section{Background}

\subsection{Orbit Determination Process}
The orbit determination process is, at the fundamental level, one which determines a celestial body's motion relative to another. Typically, this process is used to determine the motion of Earth-launched satellites relative to Earth. Though the problem can and is often applied to a variety of systems, but this paper will concern itself with Earth-centered satellites and their dynamical states. 

The state of a satellite is "a set of parameters required to predict future motion of the system"\cite{tapley2004statistical}. These parameters include the position and velocity vectors of the satellite, and often includes other information relating to the dynamical model. Other information can include atmospheric drag, solar wind, gravity terms, tracking station information, or other system dynamics. Fundamentally, anything can be included in the state that the operator would wish to track and model. 

The process of determining a satellite state at a given Epoch involves convolving information about its present and past state, in a mathematically optimized manner. Present state information comes from both physical observations of the system, as well as from a dynamical system model. Observations often comes from range, range-rate, azimuth, elevation, angel, and other physically observable quantities, provided by tracking ground station or other celestial bodies. The dynamical model is a purely mathematical approximation of the satellite's state in time. Information about the satellite's past state (\emph{a-priori}) information) comes from the navigator's historical data. 

Important to note is the fact that all information about the state is imperfect. Observations have biases and accuracy in measurements, any model will have errors and unknown factors at play, and \emph{a-priori} information is a result of similar imperfections obtained from past information. Basically, the true state of the satellite is never known. The OD process is one which uses all of the imperfect information to generate a statistical "best" estimate of the satellite state at a given epoch. 

Another key note is that the relation between observation of the state and the state itself are highly nonlinear in most applications. 

\subsection{The Batch Processor}
The Batch Processor is one formulation of the OD process. 


%----------------------------------------------------------------------------------------------------------------------------------------



\section{My system}








\newpage
\bibliographystyle{abbrv}
\bibliography{report}

%# To get citations to work, I had to run:
%>>>pdflatex ASEN5070Project.tex'
%twice from command line. Don't know why.... but it works out
%
%But I fixed it in the preferences!!! Preview-->Bibtex-->number of runs:2. Why is this?





\end{document}