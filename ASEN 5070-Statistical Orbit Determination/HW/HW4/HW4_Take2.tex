\documentclass[]{article}

\begin{document}

\title{ASEN 5005-Statistical Orbit Determination
Homework 4}
\author{Zach Dischner}
\date{9/24/2012}
\maketitle


\section{Problem 1}
Given the joint density function:

\[\begin{array}[b]{ccc}
f(x,y)=k*(x^2+y^2)   &  0<x<2,&1\le y\le 3 \cr 
 f(x,y) = 0 & elsewhere & 
\end{array}\]

Several insights were to be found.


%%%%%%%%%%%%%%%%%%%%%%  << 1a >>  %%%%%%%%%%%%%%%%%%%%%%%%

\subsection{1a-Find k} 
To find {\bf k}, I employed the rule that any joint density function must be equal to one when integrated across the number range. 


\begin{equation} 
 \int_{-\infty}^\infty{\int_{-\infty}^\infty{ k*(x^2 + y^2) } dx dy} = 1 
\end{equation}

\noindent Since the validity of the function was limited to a specific number range for each variable, the joint density function  becomes:

\begin{equation}
\int_1^3{\int_0^2{ k*(x^2 + y^2) } dx dy} = 1 
\end{equation}

\noindent First, I integrated with respect to {\bf x}

\begin{equation}
{k*\int_1^3{(\frac{x^3 }{ 3} + x*y^2) } \Big{|}_0^2 dy} = 1 
\end{equation}

\noindent Then after evaluating for the {\bf x} range, I integrated with respect to  {\bf y}

\begin{equation}
{k*\Big{ [ } y*\frac{8}{3} + 2*\frac{y^3}{3} \Big{ ] }  \Big{|}_1^3 dy} = 1 
\end{equation}

\noindent Which evaluates down to

\begin{equation}
k*\Big{ [ }  22.67 \Big{ ] } = 1 
\end{equation}

\noindent Yielding a final value of 

\begin{equation}
	\boxed {\LARGE k = 0.0441} 
\end{equation}

\

Cool, this is working alright. 




\end{document}
















